\documentclass[10pt,a4paper]{article}
\setlength{\columnseprule}{0.1pt}
\setlength{\columnsep}{1cm}
\pagenumbering{gobble}
\usepackage{StileFormulario}
\renewcommand{\arraystretch}{1.15}

\newcommand{\de}{{\ensuremath{ \mbox{d}}}}
\newcommand{\norm}[1]{{\ensuremath{||{#1}||}}}
\newcommand{\ang}[1]{{\ensuremath{\langle {#1}\rangle}}}
\newcommand{\dpar}[2]{{\ensuremath{\frac{\partial {#1}}{\partial {#2}}}}}
\newcommand{\Lusc}{{\ensuremath{L^{\vec{}}}}}

\newcommand{\Pa}{ \text{ Pa} }
\renewcommand{\bar}{ \text{ bar}}
\newcommand{\atm}{ \text{ atm}}
\newcommand{\Kg}{ \text{ Kg} }
\newcommand{\m}{ \text{ m} }
\newcommand{\N}{ \text{ N} }
\newcommand{\C}{ \text{ C} }
\newcommand{\J}{ \text{ J} }
\renewcommand{\cal}{ \text{ cal} }
\newcommand{\mol}{ \text{ mol} }
\newcommand{\K}{ \text{ K} }

\title{Formulario di Fisica 3}
\date{}
\author{}

\begin{document}
\maketitle

\begin{multicols}{2}
  {\bf Disclaimer}: In classe pare usi $L_\text{entrante}=L$: è positivo se comprimo il gas.
  A volte compare $L_{uscente}=\Lusc=-L$.
  
  \section{Termodinamica}
  {\vskip 1em \large\bf Tavola delle costanti}\vskip 0.5em
  \begin{tabular}{lll}
    $R$       & $8,31 \frac{\J}{\mol \K}$                                 & costante universale dei gas \\
    $N_A$     & $6,022 \cdot 10^{23} \frac{1}{\mol}$                      & numero di Avogadro          \\
    $k = k_B$ & $k_B = \frac{R}{N_A} = 1,38 \cdot 10^{-23} \frac{\J}{\K}$ & costante di Boltzmann       \\
  \end{tabular}

  \vskip 1.5em
  \begin{tabular}{lll}
    {\bf Simbolo} &                                    & {\bf Unità di Misura}                  \\
    $p$           & pressione                          & $1 \Pa = 1 \frac{\N}{\m^2}$            \\
    $V$           & volume                             &                                        \\
    $n$           & numero di moli                     & $n = \frac{N}{N_A}$                    \\
    $T$           & temperatura assoluta               & $K$                                    \\
    $C_V$         & calore molare a volume costante    & {\it SOLO PER I GAS, NON PER I SOLIDI} \\
    $C_P$         & calore molare a pressione costante &                                        \\
  \end{tabular}

  \begin{formula}[Conversioni di unità di misure]
    1 \bar = 100 000 \Pa     \\
    1 \atm = 1,013 \bar      \\
    273,15 \K = 0 \degree \C \\
    1 \cal = 4,186 \J        \\
  \end{formula}
  
\textbf{Variabili termodinamiche}\\
Dato un sistema descritto da delle variabili termodinamiche.
Date $n$ sostanze in $f$ fasi, occorrono $2+n-f$ variabili indipendenti per descrivere il sistema, che chiameremo {\it variabili termodinamiche indipendenti}.

Molto spesso useremo un gas ($f=1$) omogeneo ($n=1$), per cui necessitiamo di $2$ variabili termodinamiche indipendenti, per esempio $(p,V)$ oppure $(T,p)$. (oppure forse $(S,T)$, non sappiamo che vuol dire).

Quindi si ha che le variabili termodinamiche sono legate tra loro da alcune equazioni, dette equazioni di stato.

In pratica, se $(p,V,T)$ descrivono il sistema, deve esistere una equazione che le lega, quindi $F(p,V,T)=0$. L'espressione di $F$ è data a priori e dipende dal sistema.

Una funzione di stato è una funzione che dipende dalle variabili termodinamiche del sistema. Esempi: Entropia $S$ e Energia Interna $U$.

Quando la funzione di stato è $F(p,V,T)$ si può spesso ricavare una variabile in funzione delle altre e quindi una volta scritta $p(v,T)= cost$ con il Teorema del Dini si ricavano le relazioni tra le derivate. (In particolare vale che il prodotto ciclico è $-1$.)

In generale, nel calcolare la quantità $\frac{\partial x}{\partial y}$ supporremo fissata la quantità $z$, dove $x,y,z$ sono le nostre variabili termodinamiche.

\begin{formula}[Coefficienti]
\text{coefficiente di espansione volumica:}\\
\beta=\frac{1}{V} \left( \frac{\partial V}{\partial T} \right)_{p fisso} \\
\text{coefficiente di compressibilità isoterma:}\\
\kappa=-\frac{1}{V} \left( \frac{\partial V}{\partial p} \right)_{T fisso}

\end{formula}
\begin{formula}[Relazioni di Maxwell per gas omogeneo]
\text{anche qua l'altra variabile è fissata}\\
\frac{\partial T}{\partial p}=\frac{\kappa}{\beta} \quad
\frac{\partial T}{\partial V}=\frac{1}{\beta V} \\
\frac{\partial p}{\partial T}=\frac{\beta}{\kappa} \quad
\frac{\partial p}{\partial V}=-\frac{1}{\kappa V}\\
\frac{\partial V}{\partial T}=\beta V \quad
\frac{\partial V}{\partial p}=-\kappa V
\end{formula}
  
\begin{formula}[Lavoro su un solido]
 \de T=0 \\
 L=-\int p \de V \\
 \de V= - \kappa V \de P \\
 L=\int^{p_2}_{p_1} \kappa V p \de p \cong \frac{\kappa V}{2} (p_2^2 -p_1^2)
\end{formula}
  
  
  \begin{formula}[Primo principio della Termodinamica]
  \text{rappresenta la conservazione dell'energia}\\
  \text{definisco il calore in questo modo} \\
    \de U = \delta Q + \delta L \\
    L =- \int p \de V  \\
  \end{formula}

Per equazioni di stato generiche non è chiaro come calcolare $U$ oppure $\delta Q$. 
Vorremmo definire $C=\frac{\delta Q}{\partial T}$, ma questo dipende dalla trasformazione. Se riscaldiamo a volume costante otteniamo $C=C_V$, se riscaldiamo a pressione costante otteniamo $C=C_p$.

\begin{formula}[Calori specifici]
  C_V = \dpar{U}{T}\\
  \text{$C_V$ è il calore a volume costante}\\
  
  C_p= C_V + \left[ \dpar{U}{V} + p \right] V \beta \\
  \text{$C_p$ è il calore a pressione costante}\\
\end{formula}

Queste formule sono utili quando ci viene data una qualsiasi equazione di stato e qualsiasi energia interna. Nel caso generale $C_V$ e $C_p$ possono non essere costanti, tipicamente dipendono dalla temperatura. Nel caso dei gas perfetti sono costanti.

\begin{formula}[Secondo principio termodinamica]

\end{formula}


  \begin{formula}[Energia Interna per i gas perfetti]
    U = \frac{\ell}{2}nRT = n c_V T=C_V T
  \end{formula}

  \begin{itemize}
  \item Gas Monoatomici: $\ell =3$.
  \item Gas Biatomici: $\ell =5$.
  \item Solidi: $\ell =6$.
  \end{itemize}

  \begin{formula}[Equazione di Stato dei gas Perfetti]
    pV = nRT = NkT
  \end{formula}

  \begin{formula}[Relazione di Mayer]
    c_p - c_V = R              \\
    \gamma = \frac{c_p}{c_V}   \\
    \frac{R}{c_V} = \gamma - 1 \\
  \end{formula}

  \begin{formula}[Leggi per trasformazione generica, per gas perfetti]
    \Delta U = Q - \Lusc                                                                      \\
    U = n c_V T = \frac{1}{\gamma -1} nRT                                                          \\
    \Delta S = nc_V \log \left( \frac{T_2}{T_1} \right) + n R \log \left( \frac{V_2}{V_1} \right)  \\
    \Delta S = n c_p \log \left( \frac{T_2}{T_1} \right) - n R \log \left( \frac{p_2}{p_1} \right) \\
  \end{formula}
  
  \begin{formula}[Isobara ($p$ costante)]
    \Delta U = n c_V \Delta T                            \\
    Q = n c_p \Delta T                                   \\
    \Lusc = nR \Delta T = p \Delta V                \\
    \Delta S = n c_p \log \left( \frac{T_2}{T_1} \right) \\
  \end{formula}
	
  \begin{formula}[Isoterma ($T$ costante)]
    \Delta U = 0 \implies Q = \Lusc                 \\
    Q = nRT \log \left( \frac{V_2}{V_1} \right)          \\
    \Lusc = nRT \log \left( \frac{V_2}{V_1} \right) \\
    \Delta S = nR \log \left( \frac{V_2}{V_1} \right)    \\
  \end{formula}

  \begin{formula}[Isocora ($V$ costante)]
    \Delta U = n c_V \Delta T                            \\
    Q = n c_V \Delta T                                   \\
    \Lusc = 0                                       \\
    \Delta S = n c_V \log \left( \frac{T_2}{T_1} \right) \\
  \end{formula}

  \begin{formula}[Adiabatica reversibile senza scambio di calore]
    \Delta U = n c_V \Delta T                                          \\
    Q = 0                                                              \\
    \Lusc = \Delta U = n c_V \Delta T = \frac{\Delta (pV)}{1 - \gamma} \\
    pV^\gamma = \text{cost}                                            \\
    TV^{\gamma -1} = \text{cost}                                       \\
    pT^{\frac{\gamma}{1-\gamma}} = \text{cost}                         \\
    \text{l'entropia non varia, verificare}
  \end{formula}
  
 
  \textbf{ Espansione libera} \\
  In generale vale $$\delta Q=0, \delta L=0 \Rightarrow U=cost$$
  Nel caso dei gas perfetti, per definizione, vale $\de T=0$, cioè la temperatura, e quindi l'energia interna, non cambia.
  $\Delta U=0$.
  Prendendo un'isoterma che passa per il punto iniziale e finale si calcola che $\Delta S = nR \log \left( \frac{V_2}{V_1} \right) $


  \begin{formula}[Relazioni sui differenziali, gas perfetto]
    \de{U} + \delta \Lusc = \delta{Q}                                   \\
    pV = nRT \implies p\de{V} + V\de{p} = nR\de{T}                      \\
    \delta \Lusc  = p\de{V},                                          \\
    \de{U} = nc_V \de{T}                                                \\
    \de{U} \frac{R}{c_V} = nR\de{T} = \delta \Lusc + V\de{p} =       \\
    = \delta {Q} - \de{U} + V\de{p} \implies \delta {Q} = nc_p \de{T} - V\de{p} \\
    \text{Vale anche:}\\
    \delta Q= \de U - \delta L= C_V \de T + p \de V
  \end{formula}

  \begin{formula}[Legge di Van der Waals]
    (p+a\frac{n^2}{V^2})(V-nb) = nRT
  \end{formula}



  {\bf Fatti generici}
  \begin{itemize}
  \item Entropia per i Gas Perfetti: $S = nc_V \log T + nR \log \frac{V}{n}$
  \item Calore assorbito (per i non-gas): $\mbox{d}Q = c m \mbox{d}T$, con $c$ calore specifico del corpo.
  \item Entropia: $T \mbox{d}S = \mbox{d}U + p \mbox{d}V$, $\mbox{d}S = \left(\frac{\mbox{d}Q}{T}\right)_{\mbox{reversibile}}$
  \item Calore assorbito (per i gas): $\mbox{d}Q = n c_V \mbox{d}T$
  \item Energia libera (o potenziale di Helmholz): $F = U - TS$
  \item Entalpia: $H = U + PV$, $\Delta H < 0$ per trasformazioni spontanee.
  \item Energia libera di Gibbs: $G = H - TS$
  \end{itemize}

  \begin{itemize}
  \item Capacità Termica di un corpo: $C = \frac{\Delta Q}{\Delta T}$
  \item Conduzione: $\frac{Q}{\Delta t} = \frac{k A \Delta T}{d}$; $k$ conducibilit\`a termica, $A$ Area, $d$ spessore parete.
  \item Irraggiamento: $\frac{\de E}{\de t} = \varepsilon \sigma A (\Delta T^4)$, $\varepsilon$ emissivit\`a, $\sigma$ costante di Stefan-Boltzmann.
  \end{itemize}

  \subsection*{Ciclo di Carnot}
  Compressione Adiabatica, Espansione Isoterma, Espansione Adiabatica, Compressione Isoterma
  
  \begin{itemize}
  \item Rendimento di un ciclo: $\eta = \frac{\Lusc}{Q_{ass}}$
  \item Coefficiente di effetto frigogeno: $\mbox{COP} = \frac{Q_{\mbox{tolto al frigo}}}{L}$
  \end{itemize}

  \subsection*{``Approfondimento''}
  \begin{itemize}
  \item Legge di Dalton: "In una miscela di gas la pressione totale \`e uguale alla somma delle pressioni parziali dei suoi gas componenti". $$P_{TOT} = \frac{RT}{V}\left( \sum_{i} n_i \right)$$
  \item Forza media che {\bf una} molecola esercita sul contenitore cubico di lato $L$: $\norm{\vec{F}} = \frac{mv^2}{L}$
  \item Forza totale esercitata dal gas: $\norm{\vec{F}} = \frac{1}{3}N \left(\frac{m \ang{v^2}}{L} \right)$, con $\ang{v^2}$ valor medio del quadrato della velocit\`a. $v_{qm} := (\ang{v^2})^{\frac{1}{2}}$ \`e la velocit\`a quadratica media.
  \item $P = \frac{\norm{\vec{F_{TOT}}}}{L^2} = \frac{2}{3} N \left(\frac{1}{2}m {v_{qm}^{2}} \right) \frac{1}{V} = \frac{2}{3} N \ang{E_{cin}}$, $\ang{E_{cin}} = \frac{l}{2}K_BT = \frac{1}{2}m{v_{qm}^\frac{1}{2}}$, $l$ gradi di libert\`a.
  \end{itemize}

\end{multicols}
\end{document}

