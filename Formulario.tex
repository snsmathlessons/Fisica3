\documentclass[10pt,a4paper]{article}
\setlength{\columnseprule}{0.1pt}
\setlength{\columnsep}{1cm}
\pagenumbering{gobble}
\usepackage{StileFormulario}
\renewcommand{\arraystretch}{1.15}

\newcommand{\de}{{\ensuremath{ \mbox{d}}}}
\newcommand{\norm}[1]{{\ensuremath{||{#1}||}}}
\newcommand{\ang}[1]{{\ensuremath{\langle {#1}\rangle}}}
\newcommand{\dpar}[2]{{\ensuremath{\frac{\partial {#1}}{\partial {#2}}}}}
\newcommand{\Lusc}{{\ensuremath{L^{\vec{}}}}}

\newcommand{\Pa}{ \text{ Pa} }
\renewcommand{\bar}{ \text{ bar}}
\newcommand{\atm}{ \text{ atm}}
\newcommand{\Kg}{ \text{ Kg} }
\newcommand{\m}{ \text{ m} }
\newcommand{\N}{ \text{ N} }
\newcommand{\C}{ \text{ C} }
\newcommand{\J}{ \text{ J} }
\renewcommand{\cal}{ \text{ cal} }
\newcommand{\mol}{ \text{ mol} }
\newcommand{\K}{ \text{ K} }

\title{Formulario di Fisica 3}
\date{}
\author{}

\begin{document}
\maketitle

\begin{multicols}{2}
  {\bf Disclaimer}: In classe pare usi $L_\text{entrante}=L$: è positivo se comprimo il gas.
  A volte compare $L_{uscente}=\Lusc=-L$.
  
  \section{Termodinamica}
  {\vskip 1em \large\bf Tavola delle costanti}\vskip 0.5em
  \begin{tabular}{lll}
    $R$       & $8,31 \frac{\J}{\mol \K}$                                 & costante universale dei gas \\
    $N_A$     & $6,022 \cdot 10^{23} \frac{1}{\mol}$                      & numero di Avogadro          \\
    $k = k_B$ & $k_B = \frac{R}{N_A} = 1,38 \cdot 10^{-23} \frac{\J}{\K}$ & costante di Boltzmann       \\
  \end{tabular}

  \vskip 1.5em
  \begin{tabular}{lll}
    {\bf Simbolo} &                                    & {\bf Unità di Misura}                  \\
    $p$           & pressione                          & $1 \Pa = 1 \frac{\N}{\m^2}$            \\
    $V$           & volume                             &                                        \\
    $n$           & numero di moli                     & $n = \frac{N}{N_A}$                    \\
    $T$           & temperatura assoluta               & $K$                                    \\
    $C_V$         & calore molare a volume costante    & {\it SOLO PER I GAS, NON PER I SOLIDI} \\
    $C_P$         & calore molare a pressione costante &                                        \\
  \end{tabular}

  \begin{formula}[Conversioni di unità di misure]
    1 \bar = 100 000 \Pa     \\
    1 \atm = 1,013 \bar      \\
    273,15 \K = 0 \degree \C \\
    1 \cal = 4,186 \J        \\
  \end{formula}
  
\textbf{Variabili termodinamiche}\\
Dato un sistema descritto da delle variabili termodinamiche.
Date $n$ sostanze in $f$ fasi, occorrono $2+n-f$ variabili indipendenti per descrivere il sistema, che chiameremo {\it variabili termodinamiche indipendenti}.

Molto spesso useremo un gas ($f=1$) omogeneo ($n=1$), per cui necessitiamo di $2$ variabili termodinamiche indipendenti, per esempio $(p,V)$ oppure $(T,p)$. (oppure forse $(S,T)$, non sappiamo che vuol dire).

Quindi si ha che le variabili termodinamiche sono legate tra loro da alcune equazioni, dette equazioni di stato.

In pratica, se $(p,V,T)$ descrivono il sistema, deve esistere una equazione che le lega, quindi $F(p,V,T)=0$. L'espressione di $F$ è data a priori e dipende dal sistema.

Una funzione di stato è una funzione che dipende dalle variabili termodinamiche del sistema. Esempi: Entropia $S$ e Energia Interna $U$.

Quando la funzione di stato è $F(p,V,T)$ si può spesso ricavare una variabile in funzione delle altre e quindi una volta scritta $p(v,T)= cost$ con il Teorema del Dini si ricavano le relazioni tra le derivate. (In particolare vale che il prodotto ciclico è $-1$.)

In generale, nel calcolare la quantità $\frac{\partial x}{\partial y}$ supporremo fissata la quantità $z$, dove $x,y,z$ sono le nostre variabili termodinamiche.

\begin{formula}[Coefficienti]
\text{coefficiente di espansione volumica:}\\
\beta=\frac{1}{V} \left( \frac{\partial V}{\partial T} \right)_{p fisso} \\
\text{coefficiente di compressibilità isoterma:}\\
\kappa=-\frac{1}{V} \left( \frac{\partial V}{\partial p} \right)_{T fisso}

\end{formula}
\begin{formula}[Relazioni di Maxwell per gas omogeneo]
\text{anche qua l'altra variabile è fissata}\\
\frac{\partial T}{\partial p}=\frac{\kappa}{\beta} \quad
\frac{\partial T}{\partial V}=\frac{1}{\beta V} \\
\frac{\partial p}{\partial T}=\frac{\beta}{\kappa} \quad
\frac{\partial p}{\partial V}=-\frac{1}{\kappa V}\\
\frac{\partial V}{\partial T}=\beta V \quad
\frac{\partial V}{\partial p}=-\kappa V
\end{formula}
  
\begin{formula}[Lavoro su un solido]
 \de T=0 \\
 L=-\int p \de V \\
 \de V= - \kappa V \de P \\
 L=\int^{p_2}_{p_1} \kappa V p \de p \cong \frac{\kappa V}{2} (p_2^2 -p_1^2)
\end{formula}
  
  
  \begin{formula}[Primo principio della Termodinamica]
  \text{rappresenta la conservazione dell'energia}\\
  \text{definisco il calore in questo modo} \\
    \de U = \delta Q + \delta L \\
    L =- \int p \de V  \\
  \end{formula}

Per equazioni di stato generiche non è chiaro come calcolare $U$ oppure $\delta Q$. \\
Vorremmo definire la capacità termica $C=\frac{\delta Q}{\de T}$, ma questa dipende dalla trasformazione. Se riscaldiamo a volume costante otteniamo $C=C_V$, se riscaldiamo a pressione costante otteniamo $C=C_p$.

\begin{formula}[Capacità termica]
  C_V = \dpar{U}{T}\\
  \text{$C_V$ è la capacità termica a volume costante}\\
  
  C_p= C_V + \left[ \dpar{U}{V} + p \right] V \beta \\
  \text{$C_p$ è la capacità termica a pressione costante}\\
\end{formula}

Queste formule sono utili quando ci viene data una qualsiasi equazione di stato e qualsiasi energia interna. Nel caso generale $C_V$ e $C_p$ possono non essere costanti, tipicamente dipendono dalla temperatura. Nel caso dei gas perfetti sono costanti.

\textbf{Principio zero della termodinamica} \\
Se $A$ è in equilibrio con $B$ e $B$ è in equilibrio con $C$ allora $A$ è in equilibrio con $C$.
In cui con equilibrio termodinamico si intender equilibrio termico, meccanico, chimico, ecc.

\textbf{Secondo principio termodinamica} \\
Enunciato secondo \textbf{Kelvin-Planck}: è impossibile costruire una macchina termica che, operando in un ciclo, produca il solo effetto di convertire calore in un equivalente quantitativo di lavoro.

Enunciato secondo \textbf{Clausius}: è impossibile costruire una macchina frigorifera che, operando in un ciclo, produca il solo effetto di trasferire calore da una sorgente a temperatura più bassa a una a temperature più alta.

Enunciato secondo \textbf{Carnot}: L'efficiennza di un Ciclo di Carnot reversibile dipende solo dalle temperature delle due sorgenti tra cui opera. Un macchina di Carnot rappresenta una tra le macchine termiche più efficiente tra quelle operanti tra sorgenti a temperature fissate.

Enunciato secondo \textbf{Caratheodory}: Per ogni condizione iniziale $(p,V,T)$ e per ogni intorno (di $\bbR^3$) di questa condizione esiste un punto non raggiungibile dal punto iniziale tramite una trasformazione adiabatica reversibile. Questo è equivalente a $\delta Q= T dS$.

Enunciato "moderno": In un sistema isolato vale $\frac{\de S}{\de t} \ge 0$, $t$ è il tempo.


\textbf{Trasformazione termodinamica reversibile} \\
Una trasformazione termodinamica reversibile è una successione di trasformazioni infinitesime che passa istante per istante per degli stati di equilibrio e tale che la trasformazione inversa riporti il sistema e anche l'esterno nello stato iniziale.
In pratica si può disegnare sul piano $(V,p)$.

Esiste una trasformazione quasi-statica ma non reversibile.

\textbf{Entropia} \\
Come conseguenza del secondo principio si ha che $$ _{_{_R}} \oint \frac{\delta Q}{T}=0$$. \\
In cui con la $R$ si intende che l'integrale è calcolato lungo una trasformazione reversibile, ovvero lungo un cammino sul piano $(V,p)$.
Quindi si può definire una funzione di stato $S$ tale che per ogni coppia di stati $A$ e $B$ valga $$S(B)-S(A)= _{_{_R}}\int_{A\rightarrow B}  \frac{\delta Q}{T}$$





  \begin{formula}[Energia Interna per i gas perfetti]
    U = \frac{\ell}{2}nRT = n c_V T=C_V T
  \end{formula}

  \begin{itemize}
  \item Gas Monoatomici: $\ell =3$.
  \item Gas Biatomici: $\ell =5$.
  \item Solidi: $\ell =6$.
  \end{itemize}

  \begin{formula}[Equazione di Stato dei gas Perfetti]
    pV = nRT = NkT
  \end{formula}

  \begin{formula}[Relazione di Mayer]
    c_p - c_V = R              \\
    \gamma = \frac{c_p}{c_V}   \\
    \frac{R}{c_V} = \gamma - 1 \\
  \end{formula}
  $c_p$ è la capacità termica molare a pressione costante, anche detto calore molare a pressione costante, ovvero $\frac{C_p}{n}$, in cui $n$ è il numero di moli. $c_V$ è analogo.

  \begin{formula}[Leggi per trasformazione generica, per gas perfetti]
    \Delta U = Q - \Lusc                                                                      \\
    U = n c_V T = \frac{1}{\gamma -1} nRT                                                          \\
    \Delta S = nc_V \log \left( \frac{T_2}{T_1} \right) + n R \log \left( \frac{V_2}{V_1} \right)  \\
    \Delta S = n c_p \log \left( \frac{T_2}{T_1} \right) - n R \log \left( \frac{p_2}{p_1} \right) \\
  \end{formula}
  
  \begin{formula}[Isobara ($p$ costante)]
    \Delta U = n c_V \Delta T                            \\
    Q = n c_p \Delta T                                   \\
    \Lusc = nR \Delta T = p \Delta V                \\
    \Delta S = n c_p \log \left( \frac{T_2}{T_1} \right) \\
  \end{formula}
	
  \begin{formula}[Isoterma ($T$ costante)]
    \Delta U = 0 \implies Q = \Lusc                 \\
    Q = nRT \log \left( \frac{V_2}{V_1} \right)          \\
    \Lusc = nRT \log \left( \frac{V_2}{V_1} \right) \\
    \Delta S = nR \log \left( \frac{V_2}{V_1} \right)    \\
  \end{formula}

  \begin{formula}[Isocora ($V$ costante)]
    \Delta U = n c_V \Delta T                            \\
    Q = n c_V \Delta T                                   \\
    \Lusc = 0                                       \\
    \Delta S = n c_V \log \left( \frac{T_2}{T_1} \right) \\
  \end{formula}

  \begin{formula}[Adiabatica reversibile senza scambio di calore]
    \Delta U = n c_V \Delta T                                          \\
    Q = 0                                                              \\
    \Lusc = \Delta U = n c_V \Delta T = \frac{\Delta (pV)}{1 - \gamma} \\
    pV^\gamma = \text{cost}                                            \\
    TV^{\gamma -1} = \text{cost}                                       \\
    pT^{\frac{\gamma}{1-\gamma}} = \text{cost}                         \\
    \text{l'entropia non varia, verificare}
  \end{formula}
  
 
  \textbf{ Espansione libera} \\
  In generale vale $$\delta Q=0, \delta L=0 \Rightarrow U=cost$$
  Nel caso dei gas perfetti, per definizione, vale $\de T=0$, cioè la temperatura, e quindi l'energia interna, non cambia.
  $\Delta U=0$. \\
  Per calcolare la variazione di entropia non posso usare la formula $\int_{A\rightarrow B} \frac{\delta Q}{T}$ calcolata lungo l'espansione libera (infatti l'integrale fa $0$, poiché $\de Q=0$) poiché non è reversibile. Prendo quindi un'isoterma (reversibile!) che passa per il punto iniziale e finale si calcola che $\Delta S = nR \log \left( \frac{V_2}{V_1} \right)$


  \begin{formula}[Relazioni sui differenziali, gas perfetto]
    \de{U} + \delta \Lusc = \delta{Q}                                   \\
    pV = nRT \implies p\de{V} + V\de{p} = nR\de{T}                      \\
    \delta \Lusc  = p\de{V},                                          \\
    \de{U} = nc_V \de{T}                                                \\
    \de{U} \frac{R}{c_V} = nR\de{T} = \delta \Lusc + V\de{p} =       \\
    = \delta {Q} - \de{U} + V\de{p} \implies \delta {Q} = nc_p \de{T} - V\de{p} \\
    \text{Vale anche:}\\
    \delta Q= \de U - \delta L= C_V \de T + p \de V
  \end{formula}

  \begin{formula}[Legge di Van der Waals]
    (p+a\frac{n^2}{V^2})(V-nb) = nRT
  \end{formula}



  {\bf Fatti generici}
  \begin{itemize}
  \item Entropia per i Gas Perfetti: $S = nc_V \log T + nR \log \frac{V}{n}$
  \item Calore assorbito (per i non-gas): $\mbox{d}Q = c m \mbox{d}T$, con $c$ calore specifico del corpo.
  \item Entropia: $T \mbox{d}S = \mbox{d}U + p \mbox{d}V$, $\mbox{d}S = \left(\frac{\mbox{d}Q}{T}\right)_{\mbox{reversibile}}$
  \item Calore assorbito (per i gas): $\mbox{d}Q = n c_V \mbox{d}T$, con $c_V$ capacità termica molare a volume costante, anche detto calore molare a pressione costante. Attento che può dipendere dal libro.
  \item Energia libera (o potenziale di Helmholz): $F = U - TS$
  \item Entalpia: $H = U + pV$, $\Delta H < 0$ per trasformazioni spontanee.
  \item Energia libera di Gibbs: $G = H - TS$
  \end{itemize}

  \begin{itemize}
  \item Capacità Termica di un corpo: $C = \frac{\Delta Q}{\Delta T}$
  \item Conduzione: $\frac{Q}{\Delta t} = \frac{k A \Delta T}{d}$; $k$ conducibilit\`a termica, $A$ Area, $d$ spessore parete.
  \item Irraggiamento: $\frac{\de E}{\de t} = \varepsilon \sigma A (\Delta T^4)$, $\varepsilon$ emissivit\`a, $\sigma$ costante di Stefan-Boltzmann.
  \end{itemize}



\textbf{Ciclo di Carnot} \\
Compressione Adiabatica, Espansione Isoterma, Espansione Adiabatica, Compressione Isoterma. \\
Compressione $=$ volume diminuisce (dal latino comprimére). \\
$\eta = 1- \frac{T_L}{T_H}$

\textbf{Ciclo di Stirling} \\
Isocora ($p$ aumenta), Espansione Isoterma, Isocora ($p$ diminuisce), Compressione Isoterma. \\
$\eta = 1- \frac{T_L}{T_H}$

\textbf{Ciclo Otto, Gasoline} \\
Isocora ($p$ aumenta) (da $T_2$ a $T_3$), Espansione adiabatica, Isocora ($p$ diminuisce), Compressione Isocora. \\
$\eta = 1- \frac{T_1}{T_2}$ (da Wikipedia, Il Picasso NON è d'accordo, da controllare)

\textbf{Ciclo Diesel} \\
Compressione adiabatica (da $T_1$ a $T_2$), Espansione Isobara, Espansione Adiabatica, Isobara ($p$ diminuisce). \\
$\eta= 1-\frac{1}{\gamma} \frac{r^\gamma -1}{r-1} \frac{T_1}{T_2}$ \\
$r=\frac{V_1}{V_3}$, $\gamma$ è quello delle relazioni di Mayer.


\textbf{Rendimento di un ciclo generico} \\
Posso sempre interpretare un ciclo reversibile come una macchina termica che opera tra infinite sorgenti, una per ogni temperatura raggiunta dal sistema durante il ciclo. Questa è un'ipotesi necessaria affinché il ciclo sia reversibile. 
Geometricamente ciò corrisponde a sezionare il ciclo con delle isoterme. 
Operativamente per calcolare il rendimento si calcola il calore netto $\delta Q(T)$ scambiato con ogni sorgente $T$. Definiamo poi $Q_{ass}=\sum_T \max\{ \delta Q(T), 0 \}$, in pratica sommiamo, sorgente per sorgente, solo i calori che entrano nella macchina.
Il lavoro è l'area racchiusa dalla curva.
  
  
  \begin{itemize}
  \item Rendimento di una macchina termica: $\eta = \frac{\Lusc}{Q_{ass}}$
  \item Efficienza di un frigorifero: $\mbox{COP}=\epsilon = \frac{Q_{\mbox{tolto al frigo}}}{L}$
  \end{itemize}

  \subsection*{``Approfondimento''}
  \begin{itemize}
  \item Equazione di Clapeyron: descrive la variazione della pressione con la temperatura della curva di equilibrio (di solito si pensa nel piano $(T,p)$) tra due fasi di una stessa sostanza (è una curva per la regola delle fasi: $n=1$, $f=2$): $$\frac{\de p}{\de T} =\frac{\lambda}{T (v_B - v_A)}$$. \\
  $\lambda$ è il calore latente (per unità di massa) di transizione a una fase all'altra. \\
  $v$ è il volume specifico delle due fasi $A$ e $B$. (il volume specifico è volume diviso massa, quindi l'inverso della densità volumica)
  \item Legge di Dalton: "In una miscela di gas la pressione totale \`e uguale alla somma delle pressioni parziali dei suoi gas componenti". $$P_{TOT} = \frac{RT}{V}\left( \sum_{i} n_i \right)$$
  \item Forza media che {\bf una} molecola esercita sul contenitore cubico di lato $L$: $\norm{\vec{F}} = \frac{mv^2}{L}$
  \item Forza totale esercitata dal gas: $\norm{\vec{F}} = \frac{1}{3}N \left(\frac{m \ang{v^2}}{L} \right)$, con $\ang{v^2}$ valor medio del quadrato della velocit\`a. $v_{qm} := (\ang{v^2})^{\frac{1}{2}}$ \`e la velocit\`a quadratica media.
  \item $P = \frac{\norm{\vec{F_{TOT}}}}{L^2} = \frac{2}{3} N \left(\frac{1}{2}m {v_{qm}^{2}} \right) \frac{1}{V} = \frac{2}{3} N \ang{E_{cin}}$, $\ang{E_{cin}} = \frac{l}{2}K_BT = \frac{1}{2}m{v_{qm}^\frac{1}{2}}$, $l$ gradi di libert\`a.
  \end{itemize}

\end{multicols}
\end{document}

