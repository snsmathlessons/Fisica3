\documentclass[10pt,a4paper]{article}
\setlength{\columnseprule}{0.1pt}
\setlength{\columnsep}{1cm}
\pagenumbering{gobble}
\usepackage{StileFormulario}

\title{Formulario di Fisica 3}
\date{}
\author{}

\begin{document}
\maketitle

\begin{multicols}{2}
  \section{Termodinamica}
  \begin{formula}[Capacità Termica]
    C = \frac{Q}{T}
  \end{formula}
  
  \begin{formula}[Caloria]
    1 \mbox{ caloria } = 4.186 \mbox{ J}
  \end{formula}

  \begin{formula}[Equazione di Stato dei gas Perfetti]
    pV = nRT
  \end{formula}
  Dove ricordiamo che $p$ è la pressione, $V$ è il volume, $n = \frac{N}{N_0}$ numero di moli, $R$ è la costante dei gas, $T$ è la temperatura assoluta.
  
\end{multicols}
\end{document}

